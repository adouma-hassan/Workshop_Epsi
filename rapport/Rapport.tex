\documentclass[12pt, openany]{report}
\usepackage[utf8]{inputenc}
\usepackage[T1]{fontenc}
\usepackage[a4paper,left=2cm,right=2cm,top=2cm,bottom=2cm]{geometry}
\usepackage{listings}
\usepackage[french]{babel}
\usepackage{libertine}
\usepackage{xcolor}
\usepackage[section]{placeins}
\usepackage[pdftex]{graphicx}
\setlength{\parindent}{0cm}
\setlength{\parskip}{1ex plus 0.5ex minus 0.2ex}
\newcommand{\hsp}{\hspace{25pt}}
\newcommand{\HRule}{\rule{\linewidth}{0.5mm}}


\begin{document}
%\tableofcontents
\begin{titlepage}
  \begin{sffamily}
    \begin{center}

      \includegraphics[scale=0.3]{image/logo_epsi.png}~\\[1.5cm]

      \textsc{\LARGE École de L’ingénierie Informatique}\\[2cm]

      \textsc{\Large Workshop : }\\[1cm]


      \HRule \\[0.4cm]
      { \huge \bfseries  L'E-SANTÉ : LE NUMÉRIQUE AU SERVICE DE LA SANTÉ \\[0.4cm] }

      \HRule \\[2cm]

      % Author and supervisor
      \begin{minipage}{0.4\textwidth}
        \begin{flushleft} \large
          Professeur \textsc{:}\\
          Mydil

        \end{flushleft}
      \end{minipage}
      \begin{minipage}{0.4\textwidth}
        \begin{flushright} \large
          {Groupe }\\
          \emph{NOM 0 \\ NOM 1 \\ NOM 2 \\ NOM3
             \\ }
        \end{flushright}
      \end{minipage}

      \vfill
      {\large Mastère 1 Informatique : 10 Octobre 2024}

    \end{center}
  \end{sffamily}
  \tableofcontents
\end{titlepage}


\section {\color{red}Introduction :}
Le site d'évaluation de l'état de bien-être a été conçu pour aider les utilisateurs à comprendre et à améliorer leur santé mentale, en particulier ceux qui peuvent éprouver des symptômes de dépression ou de stress. Ce rapport analyse les différentes fonctionnalités du site, les résultats du questionnaire d'évaluation, ainsi que les missions proposées pour favoriser le bien-être mental.

\section {\color{red} Description du Contenu:}

\subsection{\color{red}Questionnaire d'Évaluation du Bien-Être:}

Le site propose un questionnaire composé de 25 questions destinées à évaluer l'état émotionnel des utilisateurs. Les questions abordent divers aspects tels que :

\begin{itemize}
    \item La fréquence des sentiments de tristesse ou de dépression.
    \item La capacité à se concentrer sur des tâches quotidiennes.
    \item La gestion des émotions et des pensées négatives.
\end{itemize}

Chaque question utilise une échelle d'auto-évaluation, permettant à l'utilisateur de choisir une réponse parmi plusieurs options (par exemple, Rarement, Parfois, Souvent, Toujours).



\subsection{\color{red}Missions du Jour:}
Une des fonctionnalités clés du site est la section "Missions du jour", qui propose des activités simples à réaliser pour améliorer l'état de bien-être des utilisateurs. Ces missions incluent :

\begin{itemize}
    \item Faire une promenade de 30 minutes à l'extérieur.
    \item Méditer pendant 10 minutes.
    \item Écrire cinq actions pour lesquelles l'utilisateur s'est dépassé.
    \item Lire un chapitre d'un livre que l'on aime.
\end{itemize}
Ces suggestions encouragent une approche proactive pour améliorer l'humeur et réduire le stress.


\subsection{\color{red}Suggestions Personnalisées:}
Après avoir complété le questionnaire, les utilisateurs reçoivent des recommandations personnalisées basées sur leurs réponses. Cela peut inclure des conseils sur les missions à réaliser ou des actions à entreprendre pour mieux gérer le stress ou les émotions négatives.

\subsection{\color{red}Efficacité du Questionnaire:}

\subsection{\color{red}Importance des Missions Quotidiennes:}
Le questionnaire est bien structuré pour détecter les symptômes courants de la dépression. Il couvre un large éventail d'émotions et de comportements, ce qui permet d'obtenir une vue d'ensemble sur l'état de bien-être de l'utilisateur. Les résultats peuvent servir de point de départ pour identifier les domaines nécessitant une attention particulière.


\section{\color{red}Impact et Bienfaits Potentiels:}
Les missions quotidiennes sont cruciales car elles fournissent des actions concrètes et réalisables que les utilisateurs peuvent intégrer dans leur routine. Ces activités physiques, mentales et sociales visent à créer des habitudes positives et à améliorer la qualité de vie .

\begin{itemize}
    \item Identifier rapidement leurs symptômes de dépression.
    \item Agir de manière proactive pour améliorer leur bien-être.
    \item Établir une routine quotidienne de soins personnels.
\end{itemize}

\section{\color{red} Limites et Améliorations Potentielles: }

\subsection{\color{red}Approfondissement des Recommandations:}

Le site propose des missions basiques et des recommandations générales. Cependant, il pourrait bénéficier de recommandations plus personnalisées, par exemple basées sur des algorithmes plus poussés, en fonction des réponses de l’utilisateur.

\subsection{\color{red}Suivi à Long Terme:}

Un système de suivi à long terme permettrait de mesurer les progrès des utilisateurs. Le site pourrait inclure un journal quotidien pour suivre les changements d’humeur, les émotions et l’évolution des missions réalisées.

\subsection{\color{red}Options d'Accompagnement Professionnel:}

Proposer une section de ressources professionnelles pour ceux qui ont besoin d'un soutien psychologique supplémentaire. Des informations sur comment consulter un psychologue ou utiliser des lignes d’assistance pourrait être utile pour les cas plus graves de dépression.



\section{\color{red} Conclusion}

Le site d'évaluation de l'état de bien-être offre un moyen accessible et simple pour évaluer les symptômes de dépression, proposer des actions pratiques, et encourager l'utilisateur à prendre soin de lui-même. Bien que l'outil soit utile pour une première approche de l’évaluation de la dépression, il pourrait être amélioré avec plus de personnalisation et des options pour un suivi à long terme.


\end{document}
